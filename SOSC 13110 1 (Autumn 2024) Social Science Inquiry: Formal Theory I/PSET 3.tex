\documentclass[11pt]{article}

% NOTE: Add in the relevant information to the commands below; or, if you'll be using the same information frequently, add these commands at the top of paolo-pset.tex file. 
\newcommand{\name}{Agustin Esteva}
\newcommand{\email}{aesteva@uchicago.edu}
\newcommand{\classnum}{13110}
\newcommand{\subject}{SSI: Formal Theory}
\newcommand{\instructors}{Jingyuan Qian}
\newcommand{\assignment}{Problem Set 3}
\newcommand{\semester}{Fall 2024}
\newcommand{\duedate}{2024-12-04}
\newcommand{\bA}{\mathbf{A}}
\newcommand{\bB}{\mathbf{B}}
\newcommand{\bC}{\mathbf{C}}
\newcommand{\bD}{\mathbf{D}}
\newcommand{\bE}{\mathbf{E}}
\newcommand{\bF}{\mathbf{F}}
\newcommand{\bG}{\mathbf{G}}
\newcommand{\bH}{\mathbf{H}}
\newcommand{\bI}{\mathbf{I}}
\newcommand{\bJ}{\mathbf{J}}
\newcommand{\bK}{\mathbf{K}}
\newcommand{\bL}{\mathbf{L}}
\newcommand{\bM}{\mathbf{M}}
\newcommand{\bN}{\mathbf{N}}
\newcommand{\bO}{\mathbf{O}}
\newcommand{\bP}{\mathbf{P}}
\newcommand{\bQ}{\mathbf{Q}}
\newcommand{\bR}{\mathbf{R}}
\newcommand{\bS}{\mathbf{S}}
\newcommand{\bT}{\mathbf{T}}
\newcommand{\bU}{\mathbf{U}}
\newcommand{\bV}{\mathbf{V}}
\newcommand{\bW}{\mathbf{W}}
\newcommand{\bX}{\mathbf{X}}
\newcommand{\bY}{\mathbf{Y}}
\newcommand{\bZ}{\mathbf{Z}}
\newcommand{\Var}{\text{Var}}

%% blackboard bold math capitals
\newcommand{\bbA}{\mathbb{A}}
\newcommand{\bbB}{\mathbb{B}}
\newcommand{\bbC}{\mathbb{C}}
\newcommand{\bbD}{\mathbb{D}}
\newcommand{\bbE}{\mathbb{E}}
\newcommand{\bbF}{\mathbb{F}}
\newcommand{\bbG}{\mathbb{G}}
\newcommand{\bbH}{\mathbb{H}}
\newcommand{\bbI}{\mathbb{I}}
\newcommand{\bbJ}{\mathbb{J}}
\newcommand{\bbK}{\mathbb{K}}
\newcommand{\bbL}{\mathbb{L}}
\newcommand{\bbM}{\mathbb{M}}
\newcommand{\bbN}{\mathbb{N}}
\newcommand{\bbO}{\mathbb{O}}
\newcommand{\bbP}{\mathbb{P}}
\newcommand{\bbQ}{\mathbb{Q}}
\newcommand{\bbR}{\mathbb{R}}
\newcommand{\bbS}{\mathbb{S}}
\newcommand{\bbT}{\mathbb{T}}
\newcommand{\bbU}{\mathbb{U}}
\newcommand{\bbV}{\mathbb{V}}
\newcommand{\bbW}{\mathbb{W}}
\newcommand{\bbX}{\mathbb{X}}
\newcommand{\bbY}{\mathbb{Y}}
\newcommand{\bbZ}{\mathbb{Z}}

%% script math capitals
\newcommand{\sA}{\mathscr{A}}
\newcommand{\sB}{\mathscr{B}}
\newcommand{\sC}{\mathscr{C}}
\newcommand{\sD}{\mathscr{D}}
\newcommand{\sE}{\mathscr{E}}
\newcommand{\sF}{\mathscr{F}}
\newcommand{\sG}{\mathscr{G}}
\newcommand{\sH}{\mathscr{H}}
\newcommand{\sI}{\mathscr{I}}
\newcommand{\sJ}{\mathscr{J}}
\newcommand{\sK}{\mathscr{K}}
\newcommand{\sL}{\mathscr{L}}
\newcommand{\sM}{\mathscr{M}}
\newcommand{\sN}{\mathscr{N}}
\newcommand{\sO}{\mathscr{O}}
\newcommand{\sP}{\mathscr{P}}
\newcommand{\sQ}{\mathscr{Q}}
\newcommand{\sR}{\mathscr{R}}
\newcommand{\sS}{\mathscr{S}}
\newcommand{\sT}{\mathscr{T}}
\newcommand{\sU}{\mathscr{U}}
\newcommand{\sV}{\mathscr{V}}
\newcommand{\sW}{\mathscr{W}}
\newcommand{\sX}{\mathscr{X}}
\newcommand{\sY}{\mathscr{Y}}
\newcommand{\sZ}{\mathscr{Z}}


\renewcommand{\emptyset}{\O}

\newcommand{\abs}[1]{\lvert #1 \rvert}
\newcommand{\norm}[1]{\lVert #1 \rVert}
\newcommand{\sm}{\setminus}


\newcommand{\sarr}{\rightarrow}
\newcommand{\arr}{\longrightarrow}

% NOTE: Defining collaborators is optional; to not list collaborators, comment out the line below.
%\newcommand{\collaborators}{Alyssa P. Hacker (\texttt{aphacker}), Ben Bitdiddle (\texttt{bitdiddle})}

\input{paolo-pset.tex}

% NOTE: To compile a version of this pset without problems, solutions, or reflections, uncomment the relevant line below.

%\excludeversion{problem}
%\excludeversion{solution}
%\excludeversion{reflection}

\begin{document}	
	
	% Use the \psetheader command at the beginning of a pset. 
	\psetheader
\section*{Problem 1}
\begin{problem}
A group has $100$ members. Each person can choose to participate or not participate in a common project.
If n of them participate in the project, then each participant gets the benefit $p(n) = n$, and each of the
$(100 - n)$ shirkers gets the benefit $s(n) = 4 + 3n.$
\end{problem}
\begin{itemize}
    \item 
    \begin{problem}
        Write the expression for the total social payoff for the group.
    \end{problem}
    \begin{solution}
    \[P = n * n + (100-n)(4 + 3n) = -2n^2 +296n + 400.\]
    \end{solution}
    \item 
    \begin{problem}
        How many participants will result in the maximum total social payoff for the group? In other words,
calculate the optimal value of n that maximizes the expression in part (a).
    \end{problem}
    \begin{solution}
        We compute the derivative with respect to $n$ to find the maximum point when $P' = 0.$
        \[P'  = -4n +296 \implies \max(n) = 74.\]
    \end{solution}
    \item 
    \begin{problem}
        n plain English, explain intuitively why the socially optimal number of n in part (b) is difficult to
achieve.
    \end{problem}
    \begin{solution}
        It becomes exponentially harder to find people to participate in the project since the benefit to a shirker is larger than the benefit to a participant. It is easy to find a few selfless souls who want to participate but with each new one it gets harder and harder.
    \end{solution}
\end{itemize}
\newpage

\section*{Problem 2}
\begin{problem}
    In the Italian province of Tuscany, mushroom hunters search through the woods each day for the most
expensive food in the world, truffles. Consider a game played by $99$ residents of San Miniato, a town
famous for its white truffles, deciding each day whether to hunt for truffles or do some other work. If $H$
people go hunting, each will, on average, find $10 - (H/10)$ ounces of truffles, which sell for $\$200/ounce.$
Those who do not go mushroom hunting will do work that earns them $\$50$ a day.
\end{problem}
\begin{enumerate}
    \item 
    \begin{problem}
        Prove that each resident prefers to go truffle hunting if no one else does, but prefers not to go truffle
hunting if everyone else does.
    \end{problem}
    \begin{solution}
        It will suffice to show that the payoff if $H=1$ is less than when $H = 99:$
        \[P(H=1) = 200(10 - \frac{1}{10}) = \$1980> \$50\]
        \[P(H=99) = 200(10 - \frac{99}{10}) = \$20< \$50.\] 
    \end{solution}
    \item 
    \begin{problem}
        Suppose the residents decide sequentially whether to go truffle hunting (that is, resident a decides
first; upon observing a’s decision, resident b makes their decision, and this process is repeated for all
99 residents). How many residents will go truffle hunting in this case? How much total daily income
is generated in the town (including the residents who find other work)?
    \end{problem}
    \begin{solution}
        Consider that this will the point when 
        \[P(H) < 50 \implies 200(10-\frac{H}{10})< 50 \implies H < 97.5\] Thus, $97$ people will be willing to go truffle hunting an so $2$ brave souls will stay behind. Each of the $97$ people will make
        \[P(H = 97) = 200(10 - \frac{97}{10}) = 60\] and so the town will make 
        \[60(97) + 50(2) = \$5920.\]
    \end{solution}
    \item 
    \begin{problem}
        One day, the mayor of San Miniato, a benevolent dictator, decides to limit the number of people who
are allowed to hunt truffles each day. How many people should be allowed to go truffle hunting each
day in order to maximize the total town income?
    \end{problem}
    \begin{solution}
        The total town income is 
        \[T = 200H(10 - \frac{H}{10}) + 50(99-H) = -20H^2
 +1950H+4950.\] Which will be maximized when 
        \[T' = 0 \implies  -40H + 1950 \implies H = 48.75.\] Thus, the dictator should allow $49$ people to go hunting in order to maximize the income.
    \end{solution}
    \item 
    \begin{problem}
        Explain why some residents who are not selected to go truffle hunting under
the mayor’s order will have an incentive to try to sneak away and go hunting anyway. How might the
townspeople address this challenge?
    \end{problem}
    \begin{solution}
        The first part of the question is obvious by (b). An individual will make more money if he goes before more than $97$ people go. To address this, the townspeople could do the following:
        \begin{itemize}
            \item Invest in public services with the large amount of money to show the people that not going truffle hunting is beneficial to all.
            \item Rotate who gets to go truffle hunting each day in order so that everyone gets the same amount of income eventually.
            \item Throw the people go hunting without a license in jail or death penalty.
        \end{itemize}
    \end{solution}
\end{enumerate}
\end{document}