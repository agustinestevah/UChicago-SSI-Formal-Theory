\documentclass[10pt, oneside]{article} 
\usepackage{amsmath, amsthm, amssymb, calrsfs, wasysym, verbatim, bbm, color, graphics, geometry, esint}
\usepackage{float}



\geometry{tmargin=.75in, bmargin=.75in, lmargin=.75in, rmargin = .75in}  

\newcommand{\bbR}{\mathbb{R}}
\newcommand{\bbC}{\mathbb{C}}
\newcommand{\bbZ}{\mathbb{Z}}
\newcommand{\bbN}{\mathbb{N}}
\newcommand{\bbP}{\mathbb{P}}
\newcommand{\bbQ}{\mathbb{Q}}
\newcommand{\Cdot}{\boldsymbol{\cdot}}
\newcommand{\scA}{\mathscr{A}}
\newcommand{\curl}{\text{curl}}

\theoremstyle{definition}
\newtheorem{exmp}{Example}[section]
\newtheorem{thm}{Theorem}
\newtheorem{defn}{Definition}
\newtheorem{prop}{Proposition}
\newtheorem{conv}{Convention}
\newtheorem{rem}{Remark}
\newtheorem{lem}{Lemma}
\newtheorem{cor}{Corollary}
\input{paolo-pset.tex}



\title{UChicago SSI: Formal Theory: 13210}
\author{Notes by Agustín Esteva, Lectures by Scott Gehlbach}
\date{Academic Year 2024-2025}

\begin{document}

\maketitle
\tableofcontents

\vspace{.25in}

\section{Lectures}

\subsection{Wednesday, Mar 24: Introduction}
In almost every strategic game, there are an odd number of Nash equilibrium.
    \begin{table}[H]
        \centering
        \begin{tabular}{c|c|c}
             & work & shirk\\
             \hline
             work& ($\theta_1 -c$,$\theta_2 - c$) & (-c,0) \\
             \hline
             shirk&  (0,-c)& (0,0)\\
        \end{tabular}
        \caption{Cooperation with Incomplete Information}
    \end{table}
Very difficult problem! I don't know what he knows what I know etc. Solution is Bayesian Nash equilibrium. If the game is sequential, then we need the perfect Bayesian equilibrium (PBE).


    We are going to be studying uncertainty. We denote the uncertainty of a player by $\theta$ or $\omega,$ and the set of possible uncertainties is $\Theta.$ 

\begin{exmp}
    Uncertainty could be a number: $\theta \in \Theta = [0,1].$ For example, Trump's true popularity in the 2020 election.

    Uncertainty could be a truth value $\theta \in \Theta = \{0,1\}.$ For example, Trump tweeting that he won the election where $\theta = 1$ iff true.
\end{exmp}

\begin{defn}
    Suppose that the uncertainty is a truth value of a statement. A \textbf{belief} is a probability $\mu = \bbP\{\theta  = 1\}.$ 

    On the other hand, if the uncertainty is a number, then the uncertainty is a cdf $F: \Theta \to [0,1].$ 
\end{defn}

\begin{defn}
    The set of all beliefs $\theta \in \Theta$ is denoted by $\Delta(\Theta),$ and it is the set of all probability distributions on $\Theta.$
\end{defn}
Note that in truth values, $\Delta(\Theta) = [0,1]$ and $\Delta([0,1]) = \{F \text{ cdf} \; | \: F: [0,1] \to [0,1]\}$

\begin{defn}
    \textbf{Information} is a process that generates signals ('data') conditional on uncertainty.
\end{defn}
\begin{exmp}
    Suppose $\theta$ is truth value on Trump's tweet in 2020. Suppose $S = \{0,1\}$ represents AP announcememnt on Trump's victory. Then we define a map $\pi: \Theta \to \Delta(S)$
    \[\pi_1 = \bbP\{s = 1 \; | \; \theta = 1\}\]
    \[\pi_0 = \bbP\{s = 1 \;  | \theta = 0\}\] $(\pi_0, \pi_1)$ represents an information, and defines an editorial policy of the AP. For example $(0,1)$ represents perfect work by the AP, always reporting correct. $(\frac{1}{2}, \frac{1}{2})$ is terrible.$(1,1)$ then AP loves Trump.
\end{exmp}
\begin{rem}
    Note that $s$ can be informative about $\theta,$ since if we suppose that $\bbP\{\theta = 1\} = \mu,$ then given that $s = 0,$ we update our belief to:
    \[\bbP\{\theta = 1 \; | \: s = 0\} = \frac{\bbP\{\theta = 1 \cap s = 0\}}{\bbP(s = 0)}= \frac{\bbP\{\theta = 1 \cap s = 0\}}{\bbP(\theta = 1 \cap s = 0) + \bbP(\theta = 0 \cap s=0)}.\] We know that 
    \[\bbP\{s = 0 | \theta = 1\} = 1-\pi_1 = \frac{\bbP(s = 0 \cap \theta = 1)}{\bbP(\theta = 1)} = \frac{\bbP(s = 0 \cap \theta = 1)}{\mu} \implies \bbP(s = 0 \cap \theta = 1) = (1-\pi_1)\mu\] Similarly for the other term, we find that 
    \[\bbP\{\theta = 1 \; | \: s = 0\} = \frac{(1-\pi_1)\mu}{(1-\pi_0)(1-\mu) + (1-\pi_1)\mu} = \frac{\mu}{\frac{1-\pi_0}{1-\pi_1}(1-\mu) + \mu},\] and so $\bbP\{\theta = 1 \; | \: s = 0\} < \mu$ if and only if $\pi_0 < \pi_1$
\end{rem}


\end{document}