\documentclass[11pt]{article}

% NOTE: Add in the relevant information to the commands below; or, if you'll be using the same information frequently, add these commands at the top of paolo-pset.tex file. 
\newcommand{\name}{Agustin Esteva}
\newcommand{\email}{aesteva@uchicago.edu}
\newcommand{\classnum}{13110}
\newcommand{\subject}{SSI: Formal Theory}
\newcommand{\instructors}{Jingyuan Qian}
\newcommand{\assignment}{Problem Set 3}
\newcommand{\semester}{Fall 2024}
\newcommand{\duedate}{2024-12-04}
\newcommand{\bA}{\mathbf{A}}
\newcommand{\bB}{\mathbf{B}}
\newcommand{\bC}{\mathbf{C}}
\newcommand{\bD}{\mathbf{D}}
\newcommand{\bE}{\mathbf{E}}
\newcommand{\bF}{\mathbf{F}}
\newcommand{\bG}{\mathbf{G}}
\newcommand{\bH}{\mathbf{H}}
\newcommand{\bI}{\mathbf{I}}
\newcommand{\bJ}{\mathbf{J}}
\newcommand{\bK}{\mathbf{K}}
\newcommand{\bL}{\mathbf{L}}
\newcommand{\bM}{\mathbf{M}}
\newcommand{\bN}{\mathbf{N}}
\newcommand{\bO}{\mathbf{O}}
\newcommand{\bP}{\mathbf{P}}
\newcommand{\bQ}{\mathbf{Q}}
\newcommand{\bR}{\mathbf{R}}
\newcommand{\bS}{\mathbf{S}}
\newcommand{\bT}{\mathbf{T}}
\newcommand{\bU}{\mathbf{U}}
\newcommand{\bV}{\mathbf{V}}
\newcommand{\bW}{\mathbf{W}}
\newcommand{\bX}{\mathbf{X}}
\newcommand{\bY}{\mathbf{Y}}
\newcommand{\bZ}{\mathbf{Z}}
\newcommand{\Var}{\text{Var}}

%% blackboard bold math capitals
\newcommand{\bbA}{\mathbb{A}}
\newcommand{\bbB}{\mathbb{B}}
\newcommand{\bbC}{\mathbb{C}}
\newcommand{\bbD}{\mathbb{D}}
\newcommand{\bbE}{\mathbb{E}}
\newcommand{\bbF}{\mathbb{F}}
\newcommand{\bbG}{\mathbb{G}}
\newcommand{\bbH}{\mathbb{H}}
\newcommand{\bbI}{\mathbb{I}}
\newcommand{\bbJ}{\mathbb{J}}
\newcommand{\bbK}{\mathbb{K}}
\newcommand{\bbL}{\mathbb{L}}
\newcommand{\bbM}{\mathbb{M}}
\newcommand{\bbN}{\mathbb{N}}
\newcommand{\bbO}{\mathbb{O}}
\newcommand{\bbP}{\mathbb{P}}
\newcommand{\bbQ}{\mathbb{Q}}
\newcommand{\bbR}{\mathbb{R}}
\newcommand{\bbS}{\mathbb{S}}
\newcommand{\bbT}{\mathbb{T}}
\newcommand{\bbU}{\mathbb{U}}
\newcommand{\bbV}{\mathbb{V}}
\newcommand{\bbW}{\mathbb{W}}
\newcommand{\bbX}{\mathbb{X}}
\newcommand{\bbY}{\mathbb{Y}}
\newcommand{\bbZ}{\mathbb{Z}}

%% script math capitals
\newcommand{\sA}{\mathscr{A}}
\newcommand{\sB}{\mathscr{B}}
\newcommand{\sC}{\mathscr{C}}
\newcommand{\sD}{\mathscr{D}}
\newcommand{\sE}{\mathscr{E}}
\newcommand{\sF}{\mathscr{F}}
\newcommand{\sG}{\mathscr{G}}
\newcommand{\sH}{\mathscr{H}}
\newcommand{\sI}{\mathscr{I}}
\newcommand{\sJ}{\mathscr{J}}
\newcommand{\sK}{\mathscr{K}}
\newcommand{\sL}{\mathscr{L}}
\newcommand{\sM}{\mathscr{M}}
\newcommand{\sN}{\mathscr{N}}
\newcommand{\sO}{\mathscr{O}}
\newcommand{\sP}{\mathscr{P}}
\newcommand{\sQ}{\mathscr{Q}}
\newcommand{\sR}{\mathscr{R}}
\newcommand{\sS}{\mathscr{S}}
\newcommand{\sT}{\mathscr{T}}
\newcommand{\sU}{\mathscr{U}}
\newcommand{\sV}{\mathscr{V}}
\newcommand{\sW}{\mathscr{W}}
\newcommand{\sX}{\mathscr{X}}
\newcommand{\sY}{\mathscr{Y}}
\newcommand{\sZ}{\mathscr{Z}}


\renewcommand{\emptyset}{\O}

\newcommand{\abs}[1]{\lvert #1 \rvert}
\newcommand{\norm}[1]{\lVert #1 \rVert}
\newcommand{\sm}{\setminus}


\newcommand{\sarr}{\rightarrow}
\newcommand{\arr}{\longrightarrow}

% NOTE: Defining collaborators is optional; to not list collaborators, comment out the line below.
%\newcommand{\collaborators}{Alyssa P. Hacker (\texttt{aphacker}), Ben Bitdiddle (\texttt{bitdiddle})}

\input{paolo-pset.tex}

% NOTE: To compile a version of this pset without problems, solutions, or reflections, uncomment the relevant line below.

%\excludeversion{problem}
%\excludeversion{solution}
%\excludeversion{reflection}

\begin{document}	
	
	% Use the \psetheader command at the beginning of a pset. 
	\psetheader
\section*{Problem 1}
\begin{problem}
    Write down five $2\times 2$ matrix games, where the five games have 0, 1, 2, 3, and 4 Nash
equilibria, respectively.
\end{problem}

\begin{solution}
   No Dash Equilibrium:
    \[
\begin{tabular}{lcc}
& $C$ & $D$ \\
\cline{2-3}
$A$ & \multicolumn{1}{|c}{$1,-1$} & \multicolumn{1}{|c|}{$-1,1$} \\
\cline{2-3}
$B$ & \multicolumn{1}{|c}{$-1,1$} & \multicolumn{1}{|c|}{$1,-1$} \\
\cline{2-3}
\end{tabular}
\]

One Nequilibrium:
\[
\begin{tabular}{lcc}
& $C$ & $D$ \\
\cline{2-3}
$A$ & \multicolumn{1}{|c}{$2,2$} & \multicolumn{1}{|c|}{$1,1$} \\
\cline{2-3}
$B$ & \multicolumn{1}{|c}{$1,1$} & \multicolumn{1}{|c|}{$0,0$} \\
\cline{2-3}
\end{tabular}
\]

Two Nequilibriums:
\[
\begin{tabular}{lcc}
& $C$ & $D$ \\
\cline{2-3}
$A$ & \multicolumn{1}{|c}{$2,2$} & \multicolumn{1}{|c|}{$0,0$} \\
\cline{2-3}
$B$ & \multicolumn{1}{|c}{$0,0$} & \multicolumn{1}{|c|}{$2,2$} \\
\cline{2-3}
\end{tabular}
\]
Three Nequilibriums:
\[
\begin{tabular}{lcc}
& $C$ & $D$ \\
\cline{2-3}
$A$ & \multicolumn{1}{|c}{$1,1$} & \multicolumn{1}{|c|}{$0,0$} \\
\cline{2-3}
$B$ & \multicolumn{1}{|c}{$1,1$} & \multicolumn{1}{|c|}{$1,1$} \\
\cline{2-3}
\end{tabular}
\]
Four Nequilibriums:
\[
\begin{tabular}{lcc}
& $C$ & $D$ \\
\cline{2-3}
$A$ & \multicolumn{1}{|c}{$1,1$} & \multicolumn{1}{|c|}{$1,1$} \\
\cline{2-3}
$B$ & \multicolumn{1}{|c}{$1,1$} & \multicolumn{1}{|c|}{$1,1$} \\
\cline{2-3}
\end{tabular}
\]
\end{solution}
\newpage
\section*{Problem 2}
\begin{problem}

\[
\begin{tabular}{lcc}
& $L$ & $R$ \\
\cline{2-3}
$U$ & \multicolumn{1}{|c}{$a,b$} & \multicolumn{1}{|c|}{$c,d$} \\
\cline{2-3}
$D$ & \multicolumn{1}{|c}{$0,0$} & \multicolumn{1}{|c|}{$0,0$} \\
\cline{2-3}
\end{tabular}
 \quad a,c \neq 0, \quad b\neq d\]

Find all Nash equilibria as a function of $a, b, c,$ and $d$, i.e., for each possible combination
of parameter values ($a, b, c,$ and $d$ are the parameters in this game), provide the set of
Nash equilibria.
\begin{solution}
    \[a<0, c<0, b<d \implies (D,L) \cap (D,R)\]
    \[a<0, c<0, b>d \implies (D,L) \cap (D,R)\]
    \[a<0, c>0, b<d \implies (D,L) \cap (U,R)\]
    \[a<0, c>0, b>d \implies (D,L)\]
    \[a>0, c<0, b<d \implies (D,R)\]
    \[a>0, c<0, b>d \implies (U,L) \cap (D,R)\]
    \[a>0, c>0, b<d \implies (U,R)\]
    \[a>0, c>0, b>d \implies (U,L)\]
\end{solution}
\end{problem}
\newpage

\section*{Problem 3}
\begin{problem}
    Consider a three-player strategic game with ordinal preferences, where the three players
simultaneously and independently choose a policy (number) between 0 and 10. The
policy implemented is the median of the three policies chosen. Each player has a most-
preferred policy: player 1 most prefers 0, player 2 most prefers 3, and player 3 most
prefers 10. All players strictly prefer implemented policies closer to their most-preferred
policy to those further away.
\end{problem}

\begin{enumerate}
    \item 
    \begin{problem}
        Show that it is a Nash equilibrium for each player to choose her most-preferred
policy.
    \end{problem}
    \begin{solution}
    Let $x = (0,3,10)$ be the action profile in which the players choose their most preferred policies. We consider the utility functions of players $a,$ $b,$ and $c,$ given that $y$ is the median of the tree choices, then
        \[\mu_1(y) = -y\]
        \[\mu_2(y) = -|y-3|\]
        \[\mu_3(y) = y-10\]
    For the action profile of $x,$ we have that $y_x = 3,$ and thus 
    \[\mu_1(y_x) = -3.\] Consider the other options given to player $1.$ If they choose policy $0,1,2,$ or $3,$ then $y = 3$ still and $\mu_1(y) = -3$. If player $1$ chooses any other policy $z,$ then the median will be $z$ and thus $\mu_1(z) = -z < -3.$ Thus, the best response of player $1$ are policy choices $\{0,1,2,3\}$ given the actions $x_{-1} = (3,10)$ by the other two players.

    Using similar logic, it is easy to show that the best response of player $2$ given $x_{-2} = (0,10)$ is policy $3$ and the best response of player $3$ is choosing policy choices $\{3,4,5,6,7,8,9,10\}.$ 

    Thus, since the action profile $x$ maximizes everybody utility such that for each player $i,$ we have that for any policy $y,$
    \[\mu_i(x) \geq \mu_i(y, x_{-i}),\] then $x$ is a Nash Equlibrium.
    \end{solution}
    

\item 
\begin{problem}
    Is this the only Nash equilibrium? If yes, explain why. If not, provide a counterexample.
\end{problem}
\begin{solution}
    No, it is not the only Nash equilibrium, as the action profile $x = (1,3,9)$ is also in equilibrium, as explained above.
\end{solution}

\end{enumerate}
\newpage
\begin{problem}
    Consider the following generalization of the VHS/Beta game discussed in class. There are infinitely many consumers (formally a continuum), indexed by $i$. Each consumer $i$
chooses VHS or Beta, where the payoff from each action is increasing in the proportion of consumers who have chosen that action. For concreteness, let the payoff from choosing VHS be $\alpha$ and from Beta be $2(1-\alpha)$, where $\alpha$ is the (endogenous) proportion of individuals who have chosen VHS.
\end{problem}
\begin{enumerate}
    \item 
    \begin{problem}
    Show that it is a Nash equilibrium for all consumers to choose VHS. (Hint:
Because consumers are a continuum, a deviation by any one consumer does not
change the proportion $\alpha$ of consumers who have chosen VHS.)
    \end{problem}
    \begin{solution}
    Let $a^\ast$ be the action profile where everybody chooses VHS, that is $\alpha = 1.$ Let $(a, a_{-i})$ be the action profile where $i$ chooses Beta and everybody else chooses VHS. Still, since there are infinitely many players, $\alpha =1.$ Thus, we have that for every player $i:$
    \[\mu(a^\ast) = \alpha = 1, \qquad \mu(a_i, a_{-i}^\ast) = 2(1-\alpha) = 0,\] and thus everybody choosing VHS is a Nash Equilibrium.
    \end{solution}
    \item 
    \begin{problem}
        Show that it is a Nash equilibrium for all consumers to choose Beta.
    \end{problem}
    \begin{solution}
        Let $a^\ast$ be the action profile where all the players choose Beta. Then we have that $\alpha = 0$ and thus $\mu(a^\ast) = 2.$ Then we have that if $(a_i, a_{-i})$ is the action profile where player $i$ chooses VHS, then 
        \[\mu(a^\ast) = 2, \qquad \mu((a_i, a_{-i}^\ast)) = 0,\] and thus $a^\ast$ is a Nash Equilibrium.
    \end{solution}
    \item 
    \begin{problem}
        In addition to the previous two equilibria, there is a third equilibrium in which
some consumers choose VHS and some choose Beta. Find this equilibrium by
deriving the $\alpha$ such that consumers are indifferent between VHS and Beta
    \end{problem}
    \begin{solution}
        When $\alpha = \frac{2}{3}$ (when 2/3 choose Beta, characterized by the action profile $a^\ast$), then suppose that $i$ are the consumers who choose VHS and $j$ be the consumers who choose Beta. Then 
        \[\mu_i(a^\ast)= \alpha = \frac{2}{3}, \quad \mu_j(a^\ast) = 2(1 - \alpha) = \frac{2}{3}.\] Now let $(a_i, a_{i-1}^\ast)$ be the action profile in which a VHS player deviates to Beta while everybody else keeps doing what they were doing, then we get the exact same payoffs as before (a similar reasoning if $j$ deviates), meaning that $\alpha = \frac{2}{3}$ is a weak Nash equilibrium. 
    \end{solution}
\end{enumerate}


\end{document}