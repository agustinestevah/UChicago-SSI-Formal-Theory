\documentclass[11pt]{article}

% NOTE: Add in the relevant information to the commands below; or, if you'll be using the same information frequently, add these commands at the top of paolo-pset.tex file. 
\newcommand{\name}{Agustin Esteva}
\newcommand{\email}{aesteva@uchicago.edu}
\newcommand{\classnum}{13210}
\newcommand{\subject}{SSI: Formal Theory II}
\newcommand{\instructors}{Scott Gehlbach}
\newcommand{\assignment}{Problem Set 2}
\newcommand{\semester}{Winter 2025}
\newcommand{\duedate}{2024-02-03}
\newcommand{\bA}{\mathbf{A}}
\newcommand{\bB}{\mathbf{B}}
\newcommand{\bC}{\mathbf{C}}
\newcommand{\bD}{\mathbf{D}}
\newcommand{\bE}{\mathbf{E}}
\newcommand{\bF}{\mathbf{F}}
\newcommand{\bG}{\mathbf{G}}
\newcommand{\bH}{\mathbf{H}}
\newcommand{\bI}{\mathbf{I}}
\newcommand{\bJ}{\mathbf{J}}
\newcommand{\bK}{\mathbf{K}}
\newcommand{\bL}{\mathbf{L}}
\newcommand{\bM}{\mathbf{M}}
\newcommand{\bN}{\mathbf{N}}
\newcommand{\bO}{\mathbf{O}}
\newcommand{\bP}{\mathbf{P}}
\newcommand{\bQ}{\mathbf{Q}}
\newcommand{\bR}{\mathbf{R}}
\newcommand{\bS}{\mathbf{S}}
\newcommand{\bT}{\mathbf{T}}
\newcommand{\bU}{\mathbf{U}}
\newcommand{\bV}{\mathbf{V}}
\newcommand{\bW}{\mathbf{W}}
\newcommand{\bX}{\mathbf{X}}
\newcommand{\bY}{\mathbf{Y}}
\newcommand{\bZ}{\mathbf{Z}}
\newcommand{\Var}{\text{Var}}

%% blackboard bold math capitals
\newcommand{\bbA}{\mathbb{A}}
\newcommand{\bbB}{\mathbb{B}}
\newcommand{\bbC}{\mathbb{C}}
\newcommand{\bbD}{\mathbb{D}}
\newcommand{\bbE}{\mathbb{E}}
\newcommand{\bbF}{\mathbb{F}}
\newcommand{\bbG}{\mathbb{G}}
\newcommand{\bbH}{\mathbb{H}}
\newcommand{\bbI}{\mathbb{I}}
\newcommand{\bbJ}{\mathbb{J}}
\newcommand{\bbK}{\mathbb{K}}
\newcommand{\bbL}{\mathbb{L}}
\newcommand{\bbM}{\mathbb{M}}
\newcommand{\bbN}{\mathbb{N}}
\newcommand{\bbO}{\mathbb{O}}
\newcommand{\bbP}{\mathbb{P}}
\newcommand{\bbQ}{\mathbb{Q}}
\newcommand{\bbR}{\mathbb{R}}
\newcommand{\bbS}{\mathbb{S}}
\newcommand{\bbT}{\mathbb{T}}
\newcommand{\bbU}{\mathbb{U}}
\newcommand{\bbV}{\mathbb{V}}
\newcommand{\bbW}{\mathbb{W}}
\newcommand{\bbX}{\mathbb{X}}
\newcommand{\bbY}{\mathbb{Y}}
\newcommand{\bbZ}{\mathbb{Z}}

%% script math capitals
\newcommand{\sA}{\mathscr{A}}
\newcommand{\sB}{\mathscr{B}}
\newcommand{\sC}{\mathscr{C}}
\newcommand{\sD}{\mathscr{D}}
\newcommand{\sE}{\mathscr{E}}
\newcommand{\sF}{\mathscr{F}}
\newcommand{\sG}{\mathscr{G}}
\newcommand{\sH}{\mathscr{H}}
\newcommand{\sI}{\mathscr{I}}
\newcommand{\sJ}{\mathscr{J}}
\newcommand{\sK}{\mathscr{K}}
\newcommand{\sL}{\mathscr{L}}
\newcommand{\sM}{\mathscr{M}}
\newcommand{\sN}{\mathscr{N}}
\newcommand{\sO}{\mathscr{O}}
\newcommand{\sP}{\mathscr{P}}
\newcommand{\sQ}{\mathscr{Q}}
\newcommand{\sR}{\mathscr{R}}
\newcommand{\sS}{\mathscr{S}}
\newcommand{\sT}{\mathscr{T}}
\newcommand{\sU}{\mathscr{U}}
\newcommand{\sV}{\mathscr{V}}
\newcommand{\sW}{\mathscr{W}}
\newcommand{\sX}{\mathscr{X}}
\newcommand{\sY}{\mathscr{Y}}
\newcommand{\sZ}{\mathscr{Z}}


\renewcommand{\emptyset}{\O}

\newcommand{\abs}[1]{\lvert #1 \rvert}
\newcommand{\norm}[1]{\lVert #1 \rVert}
\newcommand{\sm}{\setminus}


\newcommand{\sarr}{\rightarrow}
\newcommand{\arr}{\longrightarrow}

% NOTE: Defining collaborators is optional; to not list collaborators, comment out the line below.
%\newcommand{\collaborators}{Alyssa P. Hacker (\texttt{aphacker}), Ben Bitdiddle (\texttt{bitdiddle})}

% Copyright 2021 Paolo Adajar (padajar.com, paoloadajar@mit.edu)
% 
% Permission is hereby granted, free of charge, to any person obtaining a copy of this software and associated documentation files (the "Software"), to deal in the Software without restriction, including without limitation the rights to use, copy, modify, merge, publish, distribute, sublicense, and/or sell copies of the Software, and to permit persons to whom the Software is furnished to do so, subject to the following conditions:
%
% The above copyright notice and this permission notice shall be included in all copies or substantial portions of the Software.
% 
% THE SOFTWARE IS PROVIDED "AS IS", WITHOUT WARRANTY OF ANY KIND, EXPRESS OR IMPLIED, INCLUDING BUT NOT LIMITED TO THE WARRANTIES OF MERCHANTABILITY, FITNESS FOR A PARTICULAR PURPOSE AND NONINFRINGEMENT. IN NO EVENT SHALL THE AUTHORS OR COPYRIGHT HOLDERS BE LIABLE FOR ANY CLAIM, DAMAGES OR OTHER LIABILITY, WHETHER IN AN ACTION OF CONTRACT, TORT OR OTHERWISE, ARISING FROM, OUT OF OR IN CONNECTION WITH THE SOFTWARE OR THE USE OR OTHER DEALINGS IN THE SOFTWARE.

\usepackage{fullpage}
\usepackage{enumitem}
\usepackage{amsfonts, amssymb, amsmath,amsthm}
\usepackage{mathtools}
\usepackage[pdftex, pdfauthor={\name}, pdftitle={\classnum~\assignment}]{hyperref}
\usepackage[dvipsnames]{xcolor}
\usepackage{bbm}
\usepackage{graphicx}
\usepackage{mathrsfs}
\usepackage{pdfpages}
\usepackage{tabularx}
\usepackage{pdflscape}
\usepackage{makecell}
\usepackage{booktabs}
\usepackage{natbib}
\usepackage{caption}
\usepackage{subcaption}
\usepackage{physics}
\usepackage[many]{tcolorbox}
\usepackage{version}
\usepackage{ifthen}
\usepackage{cancel}
\usepackage{listings}
\usepackage{courier}

\usepackage{tikz}
\usepackage{istgame}

\hypersetup{
	colorlinks=true,
	linkcolor=blue,
	filecolor=magenta,
	urlcolor=blue,
}

\setlength{\parindent}{0mm}
\setlength{\parskip}{2mm}

\setlist[enumerate]{label=({\alph*})}
\setlist[enumerate, 2]{label=({\roman*})}

\allowdisplaybreaks[1]

\newcommand{\psetheader}{
	\ifthenelse{\isundefined{\collaborators}}{
		\begin{center}
			{\setlength{\parindent}{0cm} \setlength{\parskip}{0mm}
				
				{\textbf{\classnum~\semester:~\assignment} \hfill \name}
				
				\subject \hfill \href{mailto:\email}{\tt \email}
				
				Instructor(s):~\instructors \hfill Due Date:~\duedate	
				
				\hrulefill}
		\end{center}
	}{
		\begin{center}
			{\setlength{\parindent}{0cm} \setlength{\parskip}{0mm}
				
				{\textbf{\classnum~\semester:~\assignment} \hfill \name\footnote{Collaborator(s): \collaborators}}
				
				\subject \hfill \href{mailto:\email}{\tt \email}
				
				Instructor(s):~\instructors \hfill Due Date:~\duedate	
				
				\hrulefill}
		\end{center}
	}
}

\renewcommand{\thepage}{\classnum~\assignment \hfill \arabic{page}}

\makeatletter
\def\points{\@ifnextchar[{\@with}{\@without}}
\def\@with[#1]#2{{\ifthenelse{\equal{#2}{1}}{{[1 point, #1]}}{{[#2 points, #1]}}}}
\def\@without#1{\ifthenelse{\equal{#1}{1}}{{[1 point]}}{{[#1 points]}}}
\makeatother

\newtheoremstyle{theorem-custom}%
{}{}%
{}{}%
{\itshape}{.}%
{ }%
{\thmname{#1}\thmnumber{ #2}\thmnote{ (#3)}}

\theoremstyle{theorem-custom}

\newtheorem{theorem}{Theorem}
\newtheorem{lemma}[theorem]{Lemma}
\newtheorem{example}[theorem]{Example}

\newenvironment{problem}[1]{\color{black} #1}{}

\newenvironment{solution}{%
	\leavevmode\begin{tcolorbox}[breakable, colback=green!5!white,colframe=green!75!black, enhanced jigsaw] \proof[\scshape Solution:] \setlength{\parskip}{2mm}%
	}{\renewcommand{\qedsymbol}{$\blacksquare$} \endproof \end{tcolorbox}}

\newenvironment{reflection}{\begin{tcolorbox}[breakable, colback=black!8!white,colframe=black!60!white, enhanced jigsaw, parbox = false]\textsc{Reflections:}}{\end{tcolorbox}}

\newcommand{\qedh}{\renewcommand{\qedsymbol}{$\blacksquare$}\qedhere}

\definecolor{mygreen}{rgb}{0,0.6,0}
\definecolor{mygray}{rgb}{0.5,0.5,0.5}
\definecolor{mymauve}{rgb}{0.58,0,0.82}

% from https://github.com/satejsoman/stata-lstlisting
% language definition
\lstdefinelanguage{Stata}{
	% System commands
	morekeywords=[1]{regress, reg, summarize, sum, display, di, generate, gen, bysort, use, import, delimited, predict, quietly, probit, margins, test},
	% Reserved words
	morekeywords=[2]{aggregate, array, boolean, break, byte, case, catch, class, colvector, complex, const, continue, default, delegate, delete, do, double, else, eltypedef, end, enum, explicit, export, external, float, for, friend, function, global, goto, if, inline, int, local, long, mata, matrix, namespace, new, numeric, NULL, operator, orgtypedef, pointer, polymorphic, pragma, private, protected, public, quad, real, return, rowvector, scalar, short, signed, static, strL, string, struct, super, switch, template, this, throw, transmorphic, try, typedef, typename, union, unsigned, using, vector, version, virtual, void, volatile, while,},
	% Keywords
	morekeywords=[3]{forvalues, foreach, set},
	% Date and time functions
	morekeywords=[4]{bofd, Cdhms, Chms, Clock, clock, Cmdyhms, Cofc, cofC, Cofd, cofd, daily, date, day, dhms, dofb, dofC, dofc, dofh, dofm, dofq, dofw, dofy, dow, doy, halfyear, halfyearly, hh, hhC, hms, hofd, hours, mdy, mdyhms, minutes, mm, mmC, mofd, month, monthly, msofhours, msofminutes, msofseconds, qofd, quarter, quarterly, seconds, ss, ssC, tC, tc, td, th, tm, tq, tw, week, weekly, wofd, year, yearly, yh, ym, yofd, yq, yw,},
	% Mathematical functions
	morekeywords=[5]{abs, ceil, cloglog, comb, digamma, exp, expm1, floor, int, invcloglog, invlogit, ln, ln1m, ln, ln1p, ln, lnfactorial, lngamma, log, log10, log1m, log1p, logit, max, min, mod, reldif, round, sign, sqrt, sum, trigamma, trunc,},
	% Matrix functions
	morekeywords=[6]{cholesky, coleqnumb, colnfreeparms, colnumb, colsof, corr, det, diag, diag0cnt, el, get, hadamard, I, inv, invsym, issymmetric, J, matmissing, matuniform, mreldif, nullmat, roweqnumb, rownfreeparms, rownumb, rowsof, sweep, trace, vec, vecdiag, },
	% Programming functions
	morekeywords=[7]{autocode, byteorder, c, _caller, chop, abs, clip, cond, e, fileexists, fileread, filereaderror, filewrite, float, fmtwidth, has_eprop, inlist, inrange, irecode, matrix, maxbyte, maxdouble, maxfloat, maxint, maxlong, mi, minbyte, mindouble, minfloat, minint, minlong, missing, r, recode, replay, return, s, scalar, smallestdouble,},
	% Random-number functions
	morekeywords=[8]{rbeta, rbinomial, rcauchy, rchi2, rexponential, rgamma, rhypergeometric, rigaussian, rlaplace, rlogistic, rnbinomial, rnormal, rpoisson, rt, runiform, runiformint, rweibull, rweibullph,},
	% Selecting time-span functions
	morekeywords=[9]{tin, twithin,},
	% Statistical functions
	morekeywords=[10]{betaden, binomial, binomialp, binomialtail, binormal, cauchy, cauchyden, cauchytail, chi2, chi2den, chi2tail, dgammapda, dgammapdada, dgammapdadx, dgammapdx, dgammapdxdx, dunnettprob, exponential, exponentialden, exponentialtail, F, Fden, Ftail, gammaden, gammap, gammaptail, hypergeometric, hypergeometricp, ibeta, ibetatail, igaussian, igaussianden, igaussiantail, invbinomial, invbinomialtail, invcauchy, invcauchytail, invchi2, invchi2tail, invdunnettprob, invexponential, invexponentialtail, invF, invFtail, invgammap, invgammaptail, invibeta, invibetatail, invigaussian, invigaussiantail, invlaplace, invlaplacetail, invlogistic, invlogistictail, invnbinomial, invnbinomialtail, invnchi2, invnF, invnFtail, invnibeta, invnormal, invnt, invnttail, invpoisson, invpoissontail, invt, invttail, invtukeyprob, invweibull, invweibullph, invweibullphtail, invweibulltail, laplace, laplaceden, laplacetail, lncauchyden, lnigammaden, lnigaussianden, lniwishartden, lnlaplaceden, lnmvnormalden, lnnormal, lnnormalden, lnwishartden, logistic, logisticden, logistictail, nbetaden, nbinomial, nbinomialp, nbinomialtail, nchi2, nchi2den, nchi2tail, nF, nFden, nFtail, nibeta, normal, normalden, npnchi2, npnF, npnt, nt, ntden, nttail, poisson, poissonp, poissontail, t, tden, ttail, tukeyprob, weibull, weibullden, weibullph, weibullphden, weibullphtail, weibulltail,},
	% String functions 
	morekeywords=[11]{abbrev, char, collatorlocale, collatorversion, indexnot, plural, plural, real, regexm, regexr, regexs, soundex, soundex_nara, strcat, strdup, string, strofreal, string, strofreal, stritrim, strlen, strlower, strltrim, strmatch, strofreal, strofreal, strpos, strproper, strreverse, strrpos, strrtrim, strtoname, strtrim, strupper, subinstr, subinword, substr, tobytes, uchar, udstrlen, udsubstr, uisdigit, uisletter, ustrcompare, ustrcompareex, ustrfix, ustrfrom, ustrinvalidcnt, ustrleft, ustrlen, ustrlower, ustrltrim, ustrnormalize, ustrpos, ustrregexm, ustrregexra, ustrregexrf, ustrregexs, ustrreverse, ustrright, ustrrpos, ustrrtrim, ustrsortkey, ustrsortkeyex, ustrtitle, ustrto, ustrtohex, ustrtoname, ustrtrim, ustrunescape, ustrupper, ustrword, ustrwordcount, usubinstr, usubstr, word, wordbreaklocale, worcount,},
	% Trig functions
	morekeywords=[12]{acos, acosh, asin, asinh, atan, atanh, cos, cosh, sin, sinh, tan, tanh,},
	morecomment=[l]{//},
	% morecomment=[l]{*},  // `*` maybe used as multiply operator. So use `//` as line comment.
	morecomment=[s]{/*}{*/},
	% The following is used by macros, like `lags'.
	morestring=[b]{`}{'},
	% morestring=[d]{'},
	morestring=[b]",
	morestring=[d]",
	% morestring=[d]{\\`},
	% morestring=[b]{'},
	sensitive=true,
}

\lstset{ 
	backgroundcolor=\color{white},   % choose the background color; you must add \usepackage{color} or \usepackage{xcolor}; should come as last argument
	basicstyle=\footnotesize\ttfamily,        % the size of the fonts that are used for the code
	breakatwhitespace=false,         % sets if automatic breaks should only happen at whitespace
	breaklines=true,                 % sets automatic line breaking
	captionpos=b,                    % sets the caption-position to bottom
	commentstyle=\color{mygreen},    % comment style
	deletekeywords={...},            % if you want to delete keywords from the given language
	escapeinside={\%*}{*)},          % if you want to add LaTeX within your code
	extendedchars=true,              % lets you use non-ASCII characters; for 8-bits encodings only, does not work with UTF-8
	firstnumber=0,                % start line enumeration with line 1000
	frame=single,	                   % adds a frame around the code
	keepspaces=true,                 % keeps spaces in text, useful for keeping indentation of code (possibly needs columns=flexible)
	keywordstyle=\color{blue},       % keyword style
	language=Octave,                 % the language of the code
	morekeywords={*,...},            % if you want to add more keywords to the set
	numbers=left,                    % where to put the line-numbers; possible values are (none, left, right)
	numbersep=5pt,                   % how far the line-numbers are from the code
	numberstyle=\tiny\color{mygray}, % the style that is used for the line-numbers
	rulecolor=\color{black},         % if not set, the frame-color may be changed on line-breaks within not-black text (e.g. comments (green here))
	showspaces=false,                % show spaces everywhere adding particular underscores; it overrides 'showstringspaces'
	showstringspaces=false,          % underline spaces within strings only
	showtabs=false,                  % show tabs within strings adding particular underscores
	stepnumber=2,                    % the step between two line-numbers. If it's 1, each line will be numbered
	stringstyle=\color{mymauve},     % string literal style
	tabsize=2,	                   % sets default tabsize to 2 spaces
%	title=\lstname,                   % show the filename of files included with \lstinputlisting; also try caption instead of title
	xleftmargin=0.25cm
}

% NOTE: To compile a version of this pset without problems, solutions, or reflections, uncomment the relevant line below.

%\excludeversion{problem}
%\excludeversion{solution}
%\excludeversion{reflection}

\begin{document}	
	
	% Use the \psetheader command at the beginning of a pset. 
	\psetheader
\section*{Problem 1}
\begin{problem}
Consider a variant of the anarchy game in Ellingsen discussed in class. There are two players (i = 1, 2) who choose actions \( y_i \in [0, 1] \) representing weapons, with the remaining portion \( x_i = 1 - y_i \) representing food. The players' preferences are over their consumption \( c_i \), which is given by:



\[ 
c_1 = \begin{cases} 
x_1 + x_2 & \text{if } y_1 + \alpha \geq y_2 \\
0 & \text{otherwise}
\end{cases} 
\]





\[ 
c_2 = \begin{cases} 
x_1 + x_2 & \text{if } y_2 > y_1 + \alpha \\
0 & \text{otherwise}
\end{cases} 
\]



where \( \alpha \in (0, 1) \) gives player 1 an advantage in combat.
\begin{enumerate}
    \item Show that there is a Nash equilibrium of this game in which \( (y_1, y_2) = (1 - \alpha, 1) \).
    \begin{solution}
        Let $(y_1^\ast, y_2^\ast) = (1-\alpha, 1)$ represent our candidate Nash Equilibrium. Thus, we have that since $y_1^*+ \alpha = 1 = y_2^*,$ then
        \[u_1(y_1^\ast, y_2^\ast) = c_1(y_1^\ast, y_2^\ast) = x_1 + x_2 = \alpha + 0\]
        
        Consider some other action profile, $(y_1, y_2^*).$ We have two options, suppose $y_1 > 1-\alpha,$ in which case we have that 
        \[u_1(y_1, 1) = 1- y_1 < 1-(1-\alpha) = \alpha.\]
        Our only alternative profile for player $1$ is where $y_1 < 1 - \alpha.$ Thus, consider that 
        \[u_1(y_1 =1, y_2^*) = c_1(y_1, y_2^*) =  0, \qquad (y_1 + \alpha <1)\] Thus, player $1$ deviating from this $(y_1^*)$ is not profitable. 

        Consider now $(y_1^*, y_2),$ where $y_2 = c$ and $c<1$ (the only alternative profile to $y_2 = 1$,) then 
        \[u_2(y_1^*, y_2) = 0 \qquad (c \not> 1 ),\] and so deviating is not more profitable that $y_2^*.$

        Thus, we have that for $i = 1,2$
        \[u_i(y^*) \geq u_i(y_i, y_{-i}^*),\] and thus $(1-\alpha, 1)$ is Nash equilibrium.
    \end{solution}
    \item 
    Is this the unique Nash equilibrium? If so, explain why. If not, provide at least one example of another equilibrium.
    \begin{solution}
          Yes, it is unique. Suppose not, then there exists some $(y_1, y_2) \neq (1-\alpha, 1).$ If $y_1 < y_2-\alpha,$ then $c_1 = 0.$ However, player $1$ can just deviate such that $y_1 = y_2 - \alpha$ and get $c_1 >0.$ Similarly, if $y_1 > y_2 - \alpha,$ then $y_1$ can deviate to some $y_1 -\epsilon >y_2 - \alpha.$ Thus, we must have that $y_1 = y_2 - \alpha.$ and so $c_1 >0$ and $c_2 = 0.$ However, then player $2$ stands to profit by increasing production of weapons such that $y_1 < y_2 - \alpha,$ which we have already seen is not an equilibrium.
    \end{solution}

    \item 
    What is the weakest punishment that a (nonstrategic) state could impose on any actor \( i \) who chooses \( y_i > 0 \) that would ensure that \( (y_1, y_2) = (0, 0) \) is a Nash equilibrium?
    \begin{solution}
        Suppose the state imposes a punishment $\lambda_i$ on $y_i$ such that 
        \[u_i = c_i - \lambda_i y_i .\] Note that we have that 
        \[u_1(0,0) = 2, \qquad u_2(0,0) = 0.\]
        
        For any action profile $(y_1, y_2),$ we have that if $y_1 \geq y_2,$ then $y_1 + \alpha \geq y_2$ and so 
        \[c_1 = x_1 + x_2 = (1 - y_1) + (1-y_2) = 2 - (y_1 + y_2)\] and so 
        \[u_1(y_1, y_2 = 0) = 2-(y_1 + y_2) - \lambda y_1 = 2-(y_1  + y_2) - \lambda y_1 y_1 = 2-y_1(1 + \lambda y_1) <2,\] and so $u_1(y_1, 0) \leq u_1(0,0)$ Thus, player $1$ does not stand to profit from deviating from $(y_1, y_2) = (0,0)$ for any $\lambda
        y_1.$

        Now consider player $2,$ and suppose he deviates to some $y_2 >0.$ If $y_2 \leq \alpha,$ then his utility is obviously still $0,$ but if $y_2 >\alpha,$ then 
        \[u_2(0,y_2) = x_1 +x_2 - \lambda_2  y_2 = 1 + (1-y_2) - \lambda_2 y_2 = 2 - y_2 - \lambda_2 y_2 = 2-y_2(1 + \lambda_2) < 0\] for $\lambda_2 \leq \frac{2}{y_2} - 1,$ thus, $\lambda_i = \frac{2}{\alpha} - 1$ is the weakest punishment such that $(0,0)$ is the Nash equilibrium. Note that this punishment depends on how much weapons player $2$ produces, which is different from what we did in class. In class, we did it using 
    \[u_i - c_i - \lambda \mathbf{1}_{y_i>0},\] which would result in the same thing for player 1 (and by that I mean us not caring about them), and so for player $2,$ if he/she/they\footnote{I am not conforming to the follow executive order \url{https://www.nytimes.com/2025/01/31/us/politics/trump-pronouns.html}, sorry!} deviates to $y_2 = \alpha + \epsilon,$ then we have that 
    \[u_2(0,y_2) = 1 +(1-y_2) - \lambda  = 1 + (1 - (\alpha + \epsilon)) - \lambda = 2-\alpha -\epsilon -\lambda <0\] when $2 -\alpha - \epsilon < \lambda,$ and so $\lambda = 2-\alpha$ is smallest punishment available.
    \end{solution}
\end{enumerate}

\newpage
\section*{Problem 2}
\begin{problem}
Each player extracts $c_i$ $i = 1,2$ from the first period. The amount not extract, $y - c_1 - c_2,$ renews into $\sqrt{y-c_1 - c_2}$ for the second period. In the second period, the total is divided evenly between both players. 
\begin{enumerate}
    \item Write down the best response problem for player 1.
    \begin{solution}
        We have the utility function of player $1$ is given by 
        \[u_1(c_1, c_2) = \log(c_1) + \log(\frac{\sqrt{y- c_1 -c_2}}{2}),\] thus, the best response problem is to maximize this utility with respect to the player's own consumption, that is, to solve for 
        \[\arg\max_{c_1}\log(c_1) + \log(\frac{\sqrt{y- c_1 -c_2}}{2}) = \arg\max_{c_2}\log(c_1) + \log(\sqrt{y - c_1 - c_2}) - \log(2)\]
    \end{solution}
    \item 
    Show that the best response function is given by


\[
R_1(c_2) = \frac{2(y - c_2)}{3}
\]

\begin{solution}
        Solving the above problem requires us to find the critical points and setting equal to $0:$
    \[\frac{\partial}{\partial c_1}\log(c_1) + \log(\sqrt{y - c_1 - c_2}) - \log(2) = \frac{1}{c_1} + \frac{1}{\sqrt{y-c_1-c_2}}\frac{-1}{2\sqrt{y - c_1 -c_2}} = \frac{1}{c_1} - \frac{1}{2(y -c_1 -c_2)}\] Setting equal to $0:$
    \[\frac{1}{c_1} - \frac{1}{2(y -c_1 -c_2)} = 0 \iff c_1 = 2y - 2c_1 -2c_2 \iff 3c_1 = 2(y-c_2) \iff c_1 = \frac{2}{3}(y-c_1),\] thus, 
    \[R_1(c_2) = \frac{2(y - c_2)}{3}\]

\end{solution}
\item 
Compute the Nash Equilibrium.
\begin{solution}
    By symmetry, we have that 
    \[R_2(c_1) = \frac{2(y - c_1)}{3},\] and thus solving for when $R_2(c_1) = R_1(c_1),$ that is solving the system 
    \[c_1 = \frac{2}{3}(y-c_2), \qquad c_2 = \frac{2}{3}(y-c_1),\] by plugging in the first into the second:
    \[c_2 = \frac{2}{3}(y-\frac{2}{3}(y-c_2)) = \frac{2}{9}y + \frac{4}{9}c_2 \iff \frac{5}{9}c_2 = \frac{2}{9}y \iff c_2 = \frac{2}{5}y.\] Again, by symmetry, we must have that $c_1 = \frac{2}{5}y.$ Thus, our Nash equilibrium is when $(c_1, c_2) = (\frac{2}{5}y, \frac{2}{5}y).$ 
\end{solution}
\end{enumerate}
\begin{reflection}
Comparing our results, we see that renewing resources over time increases the amount people are going to take in the first period, but it it still not socially optimal.
\end{reflection}



\end{problem}

\end{problem}



\end{document}