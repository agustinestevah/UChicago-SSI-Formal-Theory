\documentclass[11pt]{article}

\usepackage{float}
% NOTE: Add in the relevant information to the commands below; or, if you'll be using the same information frequently, add these commands at the top of paolo-pset.tex file. 
\newcommand{\name}{Agustin Esteva}
\newcommand{\email}{aesteva@uchicago.edu}
\newcommand{\classnum}{13210}
\newcommand{\subject}{SSI: Formal Theory}
\newcommand{\instructors}{Scott Gehlbach}
\newcommand{\assignment}{Problem Set 2}
\newcommand{\semester}{Winter 2024}
\newcommand{\duedate}{2024-02-03}
\newcommand{\bA}{\mathbf{A}}
\newcommand{\bB}{\mathbf{B}}
\newcommand{\bC}{\mathbf{C}}
\newcommand{\bD}{\mathbf{D}}
\newcommand{\bE}{\mathbf{E}}
\newcommand{\bF}{\mathbf{F}}
\newcommand{\bG}{\mathbf{G}}
\newcommand{\bH}{\mathbf{H}}
\newcommand{\bI}{\mathbf{I}}
\newcommand{\bJ}{\mathbf{J}}
\newcommand{\bK}{\mathbf{K}}
\newcommand{\bL}{\mathbf{L}}
\newcommand{\bM}{\mathbf{M}}
\newcommand{\bN}{\mathbf{N}}
\newcommand{\bO}{\mathbf{O}}
\newcommand{\bP}{\mathbf{P}}
\newcommand{\bQ}{\mathbf{Q}}
\newcommand{\bR}{\mathbf{R}}
\newcommand{\bS}{\mathbf{S}}
\newcommand{\bT}{\mathbf{T}}
\newcommand{\bU}{\mathbf{U}}
\newcommand{\bV}{\mathbf{V}}
\newcommand{\bW}{\mathbf{W}}
\newcommand{\bX}{\mathbf{X}}
\newcommand{\bY}{\mathbf{Y}}
\newcommand{\bZ}{\mathbf{Z}}
\newcommand{\Var}{\text{Var}}

%% blackboard bold math capitals
\newcommand{\bbA}{\mathbb{A}}
\newcommand{\bbB}{\mathbb{B}}
\newcommand{\bbC}{\mathbb{C}}
\newcommand{\bbD}{\mathbb{D}}
\newcommand{\bbE}{\mathbb{E}}
\newcommand{\bbF}{\mathbb{F}}
\newcommand{\bbG}{\mathbb{G}}
\newcommand{\bbH}{\mathbb{H}}
\newcommand{\bbI}{\mathbb{I}}
\newcommand{\bbJ}{\mathbb{J}}
\newcommand{\bbK}{\mathbb{K}}
\newcommand{\bbL}{\mathbb{L}}
\newcommand{\bbM}{\mathbb{M}}
\newcommand{\bbN}{\mathbb{N}}
\newcommand{\bbO}{\mathbb{O}}
\newcommand{\bbP}{\mathbb{P}}
\newcommand{\bbQ}{\mathbb{Q}}
\newcommand{\bbR}{\mathbb{R}}
\newcommand{\bbS}{\mathbb{S}}
\newcommand{\bbT}{\mathbb{T}}
\newcommand{\bbU}{\mathbb{U}}
\newcommand{\bbV}{\mathbb{V}}
\newcommand{\bbW}{\mathbb{W}}
\newcommand{\bbX}{\mathbb{X}}
\newcommand{\bbY}{\mathbb{Y}}
\newcommand{\bbZ}{\mathbb{Z}}

%% script math capitals
\newcommand{\sA}{\mathscr{A}}
\newcommand{\sB}{\mathscr{B}}
\newcommand{\sC}{\mathscr{C}}
\newcommand{\sD}{\mathscr{D}}
\newcommand{\sE}{\mathscr{E}}
\newcommand{\sF}{\mathscr{F}}
\newcommand{\sG}{\mathscr{G}}
\newcommand{\sH}{\mathscr{H}}
\newcommand{\sI}{\mathscr{I}}
\newcommand{\sJ}{\mathscr{J}}
\newcommand{\sK}{\mathscr{K}}
\newcommand{\sL}{\mathscr{L}}
\newcommand{\sM}{\mathscr{M}}
\newcommand{\sN}{\mathscr{N}}
\newcommand{\sO}{\mathscr{O}}
\newcommand{\sP}{\mathscr{P}}
\newcommand{\sQ}{\mathscr{Q}}
\newcommand{\sR}{\mathscr{R}}
\newcommand{\sS}{\mathscr{S}}
\newcommand{\sT}{\mathscr{T}}
\newcommand{\sU}{\mathscr{U}}
\newcommand{\sV}{\mathscr{V}}
\newcommand{\sW}{\mathscr{W}}
\newcommand{\sX}{\mathscr{X}}
\newcommand{\sY}{\mathscr{Y}}
\newcommand{\sZ}{\mathscr{Z}}


\renewcommand{\emptyset}{\O}

\newcommand{\abs}[1]{\lvert #1 \rvert}
\newcommand{\norm}[1]{\lVert #1 \rVert}
\newcommand{\sm}{\setminus}


\newcommand{\sarr}{\rightarrow}
\newcommand{\arr}{\longrightarrow}

% NOTE: Defining collaborators is optional; to not list collaborators, comment out the line below.
%\newcommand{\collaborators}{Alyssa P. Hacker (\texttt{aphacker}), Ben Bitdiddle (\texttt{bitdiddle})}

\input{paolo-pset.tex}

% NOTE: To compile a version of this pset without problems, solutions, or reflections, uncomment the relevant line below.

%\excludeversion{problem}
%\excludeversion{solution}
%\excludeversion{reflection}

\begin{document}	
	
	% Use the \psetheader command at the beginning of a pset. 
	\psetheader
\section*{Problem 1}

\begin{problem}
Are the following two strategic games with von Neumann-Morgenstern preferences equivalent? Why or why not?
    

\[
    \begin{array}{c|cc}
        & O & B \\
        \hline
        O & 2, 1 & 0, 0 \\
        B & 0, 0 & 1, 2 \\
    \end{array}
    \quad
    \begin{array}{c|cc}
        & O & B \\
        \hline
        O & 1.25, -3.2 & -3.2, -1.3 \\
        B & -3.2, -1.3 & 1.2, 1.2 \\
    \end{array}
    \]
    
\end{problem}
\begin{solution}
    No.  On the table on the left, player $1$'s payoff to (B,B) is the same as the expected payoff to the lottery that yields $(O,O)$ with probability $\frac{1}{2}$ and $(O,B)$ with probability $\frac{1}{2}$ since
    \[\alpha_1(O) = \frac{1}{2}u_1(O,O) + \frac{1}{2}u_1(O,F) = 1\]
    \[\alpha'_1(B) = 0u_1(B,O) + 1u_1(B,B) = 1\]
    But in the second game, we have that 
    \[\alpha_1(O) = \frac{1}{2}u_1(O,O) + \frac{1}{2}u_1(O,B) = -.975\]
    \[\alpha'_1(B) = 0 u_1(B,O) + 1 u_1(B,B) = 1.2.\] Thus, we have that while in the first table, player $1$ is indifferent between a deterministic outcome of $(B,B)$ and a lottery, we have that in the second table, player $1$ prefers a lottery to a deterministic outcome $(B,B).$ This is a different way of seeing that vNM tables are not equivalent. 

    For the usual affine transformation way, suppose they are equivalent, then any utility in the second table can be expressed as $u_2 = au_1 + b.$ We see that $b = -3.2$ since $u_2 = au_1(O,B) + b = 0 + b = -3.2,$ but then $1.25 = au_2(O,O) - 3.2 = 2a - 3.2 \implies a = 2.225$ which is a contradiction since $1.2 = u_2(B,B) \neq 2.225 - 3.2$
\end{solution}



\newpage
\section*{Problem 2}

\begin{problem}
Kahneman and Tversky find that $b$ and $c$ are the modal choices in the following experiment:
    \begin{itemize}
        \item Treatment 1: A person has been given \$1000 and offered the choice of:
        \begin{itemize}
            \item Lottery $a$: 0.5 chance of an additional \$1000, 0.5 chance of no change in the endowment
            \item Lottery $b$: An additional \$500 for sure
        \end{itemize}
        \item Treatment 2: A person has been given \$2000 and offered the choice of:
        \begin{itemize}
            \item Lottery $c$: 0.5 chance of losing \$1000, 0.5 chance of no change in the endowment
            \item Lottery $d$: Losing \$500 for sure
        \end{itemize}
    \end{itemize}
    Are these choices consistent with von Neumann-Morgenstern preferences over lotteries over outcomes? Why or why not?
    
\end{problem}
\begin{solution}
    Suppose they are consistent, then since people prefer $a$ to $b,$ we have that 
    \[\frac{1}{2}u(2) + \frac{1}{2}u(1) < 1u(1.5)\] and since people prefer $c$ to $d,$ we have that
    \[\frac{1}{2}u(1) + \frac{1}{2}u(2) > 1u(1.5).\] Then chaining these equalities together
    \[\frac{1}{2}u(1) + \frac{1}{2}u(2) > \frac{1}{2}u(2) + \frac{1}{2}u(1),\] which is a contradiction, since these are obviously equal!
\end{solution}

\newpage
\section*{Problem 3}
\begin{problem}
For the following games, derive the best response for each player and find all mixed-strategy Nash equilibria.
    \begin{enumerate}
        \item 
        

\[
        \begin{array}{c|cc}
            & \text{Left} & \text{Right} \\
            \hline
            \text{Up} & -5, -5 & x - 15, 0 \\
            \text{Down} & 0, 15 & 10, 10 \\
        \end{array}
        \quad \text{where } x < 25
        \]
\begin{solution}
    Obviously, $(D,L)$ is a Nash Equilibrium and so the pure strategy of $(p, q) = (0,1)$ is a Nash Equilibrium. 
    
    For player 1: Consider that 
    \[u_1(U, q) = -5q + (x-15)(1-q)\]
    \[u_1(D,q) = 0q + 10(1-q) = 10 - 10q\] and so 
    \[q' = \frac{x- 25}{x-20} \implies u_1(U, q')= u_1(D, q').\] Thus, since $q <0<q'$ for all $x<25,$ we have that $u_1(D) > u_1(U)$ for any $q,$ and so player $1$ chooses $p = 0$ for any $q$ (he always goes down). 

    For player 2:
    \[u_2(L, p) = -5p + 15(1-p)\]
    \[u_2(R, p) = 0p + 10(1-p),\]
    and so 
    \[p' = \frac{1}{2} \implies u_1(L, p')= u_1(R, p'),\] and if $p< p',$ we have that player two goes left, ($q = 1$) and if $p>p',$ we have that player two goes right ($q = 0$). 

    Thus, we only have a single Nash equilibrium since the best responses coincide only at $((p^*, q^*) = (0,1) = (D,L).$
\end{solution}

        \item 
        

\[
        \begin{array}{c|cc}
            & \text{Left} & \text{Right} \\
            \hline
            \text{Up} & -5, -5 & x - 15, 0 \\
            \text{Down} & 0, 15 & 10, 10 \\
        \end{array}
        \quad \text{where } x = 25
        \]


    
\begin{solution}
    We see from above that if $x = 25,$ then $q' = 0 \implies u_1(U,q') = u_1(D, q').$ Thus, player $1$ chooses $p=1$ if $q=0$ since this guarantees that $u_1(U)\geq u_1(U).$ If $q>0,$ we see that $p=0$ since $u_1(U,q)< u_1(D,q).$ From above, and from the best response graph below, we see that this will intersect the best response at the points::
    \[(p^*, q^*) = \{(0,1), ([\frac{1}{2},1], 0\}.\]
\end{solution}

\item 
\begin{problem}
    Same game where $x>25.$
\end{problem}
\begin{solution}
    We see that if $q< \frac{x-25}{x-20},$ then $p = 1$ and if $q>\frac{x-25}{x-20},$ then $p = 0.$ Thus, we have three equilibrium, $(0,1), (1,0)$, and a new one at
    \[(p,q) = (\frac{1}{2}, \frac{x-25}{x-20}).\]
\end{solution}
    \end{enumerate}
\begin{figure}[H]
    \centering
    \includegraphics[width=0.5\linewidth]{Images/WIN_20250216_16_09_31_Pro.jpg}
    \caption{Best Response Functions for Above Problems}
\end{figure}
\end{problem}

\newpage
\section*{Problem 4}
\begin{problem}
\begin{tabular}{c|cc}
    & \text{Swerve} & \text{Straight} \\
    \hline
    \text{Swerve} & 0,0 & -1,1 \\
    \text{Straight} & 1,-1 & -10,-10 \\
\end{tabular}
\begin{enumerate}
    \item Derive the best response for each player, marking clearly in a graph of the two best responses which best response belongs to player 1, and which to player 2.
\begin{solution}
    We have that 
    \[u_1(Sw) = 0q + (-1)(1-q) = -1 + q\]
    \[u_2(St) = 1(q) + (-10)(1-q) = -10 + 11q\]
    Thus, we have that 
    \[q' = \frac{9}{10} \implies u_1(Sw) = u_1(St),\] and so if $q< \frac{9}{10},$ then $u_1(Sw) > u_1(St)$ and so $p = 1$ and if $q>\frac{9}{10},$ then $p = 0.$ By symmetry, we have that if 
    \[p< \frac{9}{10}\implies q = 1, \qquad p>\frac{9}{10} \implies q = 0.\]
\end{solution}
\begin{figure}[H]
    \centering
    \includegraphics[width=0.75\linewidth]{WIN_20250216_16_15_13_Pro.jpg}
    \caption{Best Response Functions}
\end{figure}
    \item Find the set of mixed-strategy Nash equilibria.
    \begin{solution}
        From the best response functions above, we see intersections at the Nash equilibria of:
        \[(p^*, q^*) = \{(0,1), (\frac{9}{10}, \frac{9}{10}), (1,0)\}\]
    \end{solution}
    \item Which, if any, equilibria do you find compelling as a prediction for behavior in the strategic environment modeled by this game (two status-conscious but reckless teenagers barreling down the road toward each other)?
    \end{enumerate}
    \begin{solution}
        In the long run, I actually think that the mixed strategy makes sense to me. Assuming these are indestructible teenagers who feel pain, it makes sense that they would each swerve, on average, nine out of ten times. 
    \end{solution}
\end{problem}






\end{document}