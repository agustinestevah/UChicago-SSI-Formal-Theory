\documentclass[11pt]{article}

% NOTE: Add in the relevant information to the commands below; or, if you'll be using the same information frequently, add these commands at the top of paolo-pset.tex file. 
\newcommand{\name}{Agustin Esteva}
\newcommand{\email}{aesteva@uchicago.edu}
\newcommand{\classnum}{13110}
\newcommand{\subject}{SSI: Formal Theory}
\newcommand{\instructors}{Scott Gehlbach}
\newcommand{\assignment}{Problem Set 1}
\newcommand{\semester}{Winter 2025}
\newcommand{\duedate}{2024-01-12}
\newcommand{\bA}{\mathbf{A}}
\newcommand{\bB}{\mathbf{B}}
\newcommand{\bC}{\mathbf{C}}
\newcommand{\bD}{\mathbf{D}}
\newcommand{\bE}{\mathbf{E}}
\newcommand{\bF}{\mathbf{F}}
\newcommand{\bG}{\mathbf{G}}
\newcommand{\bH}{\mathbf{H}}
\newcommand{\bI}{\mathbf{I}}
\newcommand{\bJ}{\mathbf{J}}
\newcommand{\bK}{\mathbf{K}}
\newcommand{\bL}{\mathbf{L}}
\newcommand{\bM}{\mathbf{M}}
\newcommand{\bN}{\mathbf{N}}
\newcommand{\bO}{\mathbf{O}}
\newcommand{\bP}{\mathbf{P}}
\newcommand{\bQ}{\mathbf{Q}}
\newcommand{\bR}{\mathbf{R}}
\newcommand{\bS}{\mathbf{S}}
\newcommand{\bT}{\mathbf{T}}
\newcommand{\bU}{\mathbf{U}}
\newcommand{\bV}{\mathbf{V}}
\newcommand{\bW}{\mathbf{W}}
\newcommand{\bX}{\mathbf{X}}
\newcommand{\bY}{\mathbf{Y}}
\newcommand{\bZ}{\mathbf{Z}}
\newcommand{\Var}{\text{Var}}

%% blackboard bold math capitals
\newcommand{\bbA}{\mathbb{A}}
\newcommand{\bbB}{\mathbb{B}}
\newcommand{\bbC}{\mathbb{C}}
\newcommand{\bbD}{\mathbb{D}}
\newcommand{\bbE}{\mathbb{E}}
\newcommand{\bbF}{\mathbb{F}}
\newcommand{\bbG}{\mathbb{G}}
\newcommand{\bbH}{\mathbb{H}}
\newcommand{\bbI}{\mathbb{I}}
\newcommand{\bbJ}{\mathbb{J}}
\newcommand{\bbK}{\mathbb{K}}
\newcommand{\bbL}{\mathbb{L}}
\newcommand{\bbM}{\mathbb{M}}
\newcommand{\bbN}{\mathbb{N}}
\newcommand{\bbO}{\mathbb{O}}
\newcommand{\bbP}{\mathbb{P}}
\newcommand{\bbQ}{\mathbb{Q}}
\newcommand{\bbR}{\mathbb{R}}
\newcommand{\bbS}{\mathbb{S}}
\newcommand{\bbT}{\mathbb{T}}
\newcommand{\bbU}{\mathbb{U}}
\newcommand{\bbV}{\mathbb{V}}
\newcommand{\bbW}{\mathbb{W}}
\newcommand{\bbX}{\mathbb{X}}
\newcommand{\bbY}{\mathbb{Y}}
\newcommand{\bbZ}{\mathbb{Z}}

%% script math capitals
\newcommand{\sA}{\mathscr{A}}
\newcommand{\sB}{\mathscr{B}}
\newcommand{\sC}{\mathscr{C}}
\newcommand{\sD}{\mathscr{D}}
\newcommand{\sE}{\mathscr{E}}
\newcommand{\sF}{\mathscr{F}}
\newcommand{\sG}{\mathscr{G}}
\newcommand{\sH}{\mathscr{H}}
\newcommand{\sI}{\mathscr{I}}
\newcommand{\sJ}{\mathscr{J}}
\newcommand{\sK}{\mathscr{K}}
\newcommand{\sL}{\mathscr{L}}
\newcommand{\sM}{\mathscr{M}}
\newcommand{\sN}{\mathscr{N}}
\newcommand{\sO}{\mathscr{O}}
\newcommand{\sP}{\mathscr{P}}
\newcommand{\sQ}{\mathscr{Q}}
\newcommand{\sR}{\mathscr{R}}
\newcommand{\sS}{\mathscr{S}}
\newcommand{\sT}{\mathscr{T}}
\newcommand{\sU}{\mathscr{U}}
\newcommand{\sV}{\mathscr{V}}
\newcommand{\sW}{\mathscr{W}}
\newcommand{\sX}{\mathscr{X}}
\newcommand{\sY}{\mathscr{Y}}
\newcommand{\sZ}{\mathscr{Z}}


\renewcommand{\emptyset}{\O}

\newcommand{\abs}[1]{\lvert #1 \rvert}
\newcommand{\norm}[1]{\lVert #1 \rVert}
\newcommand{\sm}{\setminus}


\newcommand{\sarr}{\rightarrow}
\newcommand{\arr}{\longrightarrow}

% NOTE: Defining collaborators is optional; to not list collaborators, comment out the line below.
%\newcommand{\collaborators}{Alyssa P. Hacker (\texttt{aphacker}), Ben Bitdiddle (\texttt{bitdiddle})}

% Copyright 2021 Paolo Adajar (padajar.com, paoloadajar@mit.edu)
% 
% Permission is hereby granted, free of charge, to any person obtaining a copy of this software and associated documentation files (the "Software"), to deal in the Software without restriction, including without limitation the rights to use, copy, modify, merge, publish, distribute, sublicense, and/or sell copies of the Software, and to permit persons to whom the Software is furnished to do so, subject to the following conditions:
%
% The above copyright notice and this permission notice shall be included in all copies or substantial portions of the Software.
% 
% THE SOFTWARE IS PROVIDED "AS IS", WITHOUT WARRANTY OF ANY KIND, EXPRESS OR IMPLIED, INCLUDING BUT NOT LIMITED TO THE WARRANTIES OF MERCHANTABILITY, FITNESS FOR A PARTICULAR PURPOSE AND NONINFRINGEMENT. IN NO EVENT SHALL THE AUTHORS OR COPYRIGHT HOLDERS BE LIABLE FOR ANY CLAIM, DAMAGES OR OTHER LIABILITY, WHETHER IN AN ACTION OF CONTRACT, TORT OR OTHERWISE, ARISING FROM, OUT OF OR IN CONNECTION WITH THE SOFTWARE OR THE USE OR OTHER DEALINGS IN THE SOFTWARE.

\usepackage{fullpage}
\usepackage{enumitem}
\usepackage{amsfonts, amssymb, amsmath,amsthm}
\usepackage{mathtools}
\usepackage[pdftex, pdfauthor={\name}, pdftitle={\classnum~\assignment}]{hyperref}
\usepackage[dvipsnames]{xcolor}
\usepackage{bbm}
\usepackage{graphicx}
\usepackage{mathrsfs}
\usepackage{pdfpages}
\usepackage{tabularx}
\usepackage{pdflscape}
\usepackage{makecell}
\usepackage{booktabs}
\usepackage{natbib}
\usepackage{caption}
\usepackage{subcaption}
\usepackage{physics}
\usepackage[many]{tcolorbox}
\usepackage{version}
\usepackage{ifthen}
\usepackage{cancel}
\usepackage{listings}
\usepackage{courier}

\usepackage{tikz}
\usepackage{istgame}

\hypersetup{
	colorlinks=true,
	linkcolor=blue,
	filecolor=magenta,
	urlcolor=blue,
}

\setlength{\parindent}{0mm}
\setlength{\parskip}{2mm}

\setlist[enumerate]{label=({\alph*})}
\setlist[enumerate, 2]{label=({\roman*})}

\allowdisplaybreaks[1]

\newcommand{\psetheader}{
	\ifthenelse{\isundefined{\collaborators}}{
		\begin{center}
			{\setlength{\parindent}{0cm} \setlength{\parskip}{0mm}
				
				{\textbf{\classnum~\semester:~\assignment} \hfill \name}
				
				\subject \hfill \href{mailto:\email}{\tt \email}
				
				Instructor(s):~\instructors \hfill Due Date:~\duedate	
				
				\hrulefill}
		\end{center}
	}{
		\begin{center}
			{\setlength{\parindent}{0cm} \setlength{\parskip}{0mm}
				
				{\textbf{\classnum~\semester:~\assignment} \hfill \name\footnote{Collaborator(s): \collaborators}}
				
				\subject \hfill \href{mailto:\email}{\tt \email}
				
				Instructor(s):~\instructors \hfill Due Date:~\duedate	
				
				\hrulefill}
		\end{center}
	}
}

\renewcommand{\thepage}{\classnum~\assignment \hfill \arabic{page}}

\makeatletter
\def\points{\@ifnextchar[{\@with}{\@without}}
\def\@with[#1]#2{{\ifthenelse{\equal{#2}{1}}{{[1 point, #1]}}{{[#2 points, #1]}}}}
\def\@without#1{\ifthenelse{\equal{#1}{1}}{{[1 point]}}{{[#1 points]}}}
\makeatother

\newtheoremstyle{theorem-custom}%
{}{}%
{}{}%
{\itshape}{.}%
{ }%
{\thmname{#1}\thmnumber{ #2}\thmnote{ (#3)}}

\theoremstyle{theorem-custom}

\newtheorem{theorem}{Theorem}
\newtheorem{lemma}[theorem]{Lemma}
\newtheorem{example}[theorem]{Example}

\newenvironment{problem}[1]{\color{black} #1}{}

\newenvironment{solution}{%
	\leavevmode\begin{tcolorbox}[breakable, colback=green!5!white,colframe=green!75!black, enhanced jigsaw] \proof[\scshape Solution:] \setlength{\parskip}{2mm}%
	}{\renewcommand{\qedsymbol}{$\blacksquare$} \endproof \end{tcolorbox}}

\newenvironment{reflection}{\begin{tcolorbox}[breakable, colback=black!8!white,colframe=black!60!white, enhanced jigsaw, parbox = false]\textsc{Reflections:}}{\end{tcolorbox}}

\newcommand{\qedh}{\renewcommand{\qedsymbol}{$\blacksquare$}\qedhere}

\definecolor{mygreen}{rgb}{0,0.6,0}
\definecolor{mygray}{rgb}{0.5,0.5,0.5}
\definecolor{mymauve}{rgb}{0.58,0,0.82}

% from https://github.com/satejsoman/stata-lstlisting
% language definition
\lstdefinelanguage{Stata}{
	% System commands
	morekeywords=[1]{regress, reg, summarize, sum, display, di, generate, gen, bysort, use, import, delimited, predict, quietly, probit, margins, test},
	% Reserved words
	morekeywords=[2]{aggregate, array, boolean, break, byte, case, catch, class, colvector, complex, const, continue, default, delegate, delete, do, double, else, eltypedef, end, enum, explicit, export, external, float, for, friend, function, global, goto, if, inline, int, local, long, mata, matrix, namespace, new, numeric, NULL, operator, orgtypedef, pointer, polymorphic, pragma, private, protected, public, quad, real, return, rowvector, scalar, short, signed, static, strL, string, struct, super, switch, template, this, throw, transmorphic, try, typedef, typename, union, unsigned, using, vector, version, virtual, void, volatile, while,},
	% Keywords
	morekeywords=[3]{forvalues, foreach, set},
	% Date and time functions
	morekeywords=[4]{bofd, Cdhms, Chms, Clock, clock, Cmdyhms, Cofc, cofC, Cofd, cofd, daily, date, day, dhms, dofb, dofC, dofc, dofh, dofm, dofq, dofw, dofy, dow, doy, halfyear, halfyearly, hh, hhC, hms, hofd, hours, mdy, mdyhms, minutes, mm, mmC, mofd, month, monthly, msofhours, msofminutes, msofseconds, qofd, quarter, quarterly, seconds, ss, ssC, tC, tc, td, th, tm, tq, tw, week, weekly, wofd, year, yearly, yh, ym, yofd, yq, yw,},
	% Mathematical functions
	morekeywords=[5]{abs, ceil, cloglog, comb, digamma, exp, expm1, floor, int, invcloglog, invlogit, ln, ln1m, ln, ln1p, ln, lnfactorial, lngamma, log, log10, log1m, log1p, logit, max, min, mod, reldif, round, sign, sqrt, sum, trigamma, trunc,},
	% Matrix functions
	morekeywords=[6]{cholesky, coleqnumb, colnfreeparms, colnumb, colsof, corr, det, diag, diag0cnt, el, get, hadamard, I, inv, invsym, issymmetric, J, matmissing, matuniform, mreldif, nullmat, roweqnumb, rownfreeparms, rownumb, rowsof, sweep, trace, vec, vecdiag, },
	% Programming functions
	morekeywords=[7]{autocode, byteorder, c, _caller, chop, abs, clip, cond, e, fileexists, fileread, filereaderror, filewrite, float, fmtwidth, has_eprop, inlist, inrange, irecode, matrix, maxbyte, maxdouble, maxfloat, maxint, maxlong, mi, minbyte, mindouble, minfloat, minint, minlong, missing, r, recode, replay, return, s, scalar, smallestdouble,},
	% Random-number functions
	morekeywords=[8]{rbeta, rbinomial, rcauchy, rchi2, rexponential, rgamma, rhypergeometric, rigaussian, rlaplace, rlogistic, rnbinomial, rnormal, rpoisson, rt, runiform, runiformint, rweibull, rweibullph,},
	% Selecting time-span functions
	morekeywords=[9]{tin, twithin,},
	% Statistical functions
	morekeywords=[10]{betaden, binomial, binomialp, binomialtail, binormal, cauchy, cauchyden, cauchytail, chi2, chi2den, chi2tail, dgammapda, dgammapdada, dgammapdadx, dgammapdx, dgammapdxdx, dunnettprob, exponential, exponentialden, exponentialtail, F, Fden, Ftail, gammaden, gammap, gammaptail, hypergeometric, hypergeometricp, ibeta, ibetatail, igaussian, igaussianden, igaussiantail, invbinomial, invbinomialtail, invcauchy, invcauchytail, invchi2, invchi2tail, invdunnettprob, invexponential, invexponentialtail, invF, invFtail, invgammap, invgammaptail, invibeta, invibetatail, invigaussian, invigaussiantail, invlaplace, invlaplacetail, invlogistic, invlogistictail, invnbinomial, invnbinomialtail, invnchi2, invnF, invnFtail, invnibeta, invnormal, invnt, invnttail, invpoisson, invpoissontail, invt, invttail, invtukeyprob, invweibull, invweibullph, invweibullphtail, invweibulltail, laplace, laplaceden, laplacetail, lncauchyden, lnigammaden, lnigaussianden, lniwishartden, lnlaplaceden, lnmvnormalden, lnnormal, lnnormalden, lnwishartden, logistic, logisticden, logistictail, nbetaden, nbinomial, nbinomialp, nbinomialtail, nchi2, nchi2den, nchi2tail, nF, nFden, nFtail, nibeta, normal, normalden, npnchi2, npnF, npnt, nt, ntden, nttail, poisson, poissonp, poissontail, t, tden, ttail, tukeyprob, weibull, weibullden, weibullph, weibullphden, weibullphtail, weibulltail,},
	% String functions 
	morekeywords=[11]{abbrev, char, collatorlocale, collatorversion, indexnot, plural, plural, real, regexm, regexr, regexs, soundex, soundex_nara, strcat, strdup, string, strofreal, string, strofreal, stritrim, strlen, strlower, strltrim, strmatch, strofreal, strofreal, strpos, strproper, strreverse, strrpos, strrtrim, strtoname, strtrim, strupper, subinstr, subinword, substr, tobytes, uchar, udstrlen, udsubstr, uisdigit, uisletter, ustrcompare, ustrcompareex, ustrfix, ustrfrom, ustrinvalidcnt, ustrleft, ustrlen, ustrlower, ustrltrim, ustrnormalize, ustrpos, ustrregexm, ustrregexra, ustrregexrf, ustrregexs, ustrreverse, ustrright, ustrrpos, ustrrtrim, ustrsortkey, ustrsortkeyex, ustrtitle, ustrto, ustrtohex, ustrtoname, ustrtrim, ustrunescape, ustrupper, ustrword, ustrwordcount, usubinstr, usubstr, word, wordbreaklocale, worcount,},
	% Trig functions
	morekeywords=[12]{acos, acosh, asin, asinh, atan, atanh, cos, cosh, sin, sinh, tan, tanh,},
	morecomment=[l]{//},
	% morecomment=[l]{*},  // `*` maybe used as multiply operator. So use `//` as line comment.
	morecomment=[s]{/*}{*/},
	% The following is used by macros, like `lags'.
	morestring=[b]{`}{'},
	% morestring=[d]{'},
	morestring=[b]",
	morestring=[d]",
	% morestring=[d]{\\`},
	% morestring=[b]{'},
	sensitive=true,
}

\lstset{ 
	backgroundcolor=\color{white},   % choose the background color; you must add \usepackage{color} or \usepackage{xcolor}; should come as last argument
	basicstyle=\footnotesize\ttfamily,        % the size of the fonts that are used for the code
	breakatwhitespace=false,         % sets if automatic breaks should only happen at whitespace
	breaklines=true,                 % sets automatic line breaking
	captionpos=b,                    % sets the caption-position to bottom
	commentstyle=\color{mygreen},    % comment style
	deletekeywords={...},            % if you want to delete keywords from the given language
	escapeinside={\%*}{*)},          % if you want to add LaTeX within your code
	extendedchars=true,              % lets you use non-ASCII characters; for 8-bits encodings only, does not work with UTF-8
	firstnumber=0,                % start line enumeration with line 1000
	frame=single,	                   % adds a frame around the code
	keepspaces=true,                 % keeps spaces in text, useful for keeping indentation of code (possibly needs columns=flexible)
	keywordstyle=\color{blue},       % keyword style
	language=Octave,                 % the language of the code
	morekeywords={*,...},            % if you want to add more keywords to the set
	numbers=left,                    % where to put the line-numbers; possible values are (none, left, right)
	numbersep=5pt,                   % how far the line-numbers are from the code
	numberstyle=\tiny\color{mygray}, % the style that is used for the line-numbers
	rulecolor=\color{black},         % if not set, the frame-color may be changed on line-breaks within not-black text (e.g. comments (green here))
	showspaces=false,                % show spaces everywhere adding particular underscores; it overrides 'showstringspaces'
	showstringspaces=false,          % underline spaces within strings only
	showtabs=false,                  % show tabs within strings adding particular underscores
	stepnumber=2,                    % the step between two line-numbers. If it's 1, each line will be numbered
	stringstyle=\color{mymauve},     % string literal style
	tabsize=2,	                   % sets default tabsize to 2 spaces
%	title=\lstname,                   % show the filename of files included with \lstinputlisting; also try caption instead of title
	xleftmargin=0.25cm
}

% NOTE: To compile a version of this pset without problems, solutions, or reflections, uncomment the relevant line below.

%\excludeversion{problem}
%\excludeversion{solution}
%\excludeversion{reflection}

\begin{document}	
	
	% Use the \psetheader command at the beginning of a pset. 
	\psetheader
\section*{Problem 1}
\begin{problem}
    Write down five $2\times 2$ matrix games, where the five games have 0, 1, 2, 3, and 4 Nash
equilibria, respectively.
\end{problem}

\begin{solution}
   No Dash Equilibrium:
    \[
\begin{tabular}{lcc}
& $C$ & $D$ \\
\cline{2-3}
$A$ & \multicolumn{1}{|c}{$1,-1$} & \multicolumn{1}{|c|}{$-1,1$} \\
\cline{2-3}
$B$ & \multicolumn{1}{|c}{$-1,1$} & \multicolumn{1}{|c|}{$1,-1$} \\
\cline{2-3}
\end{tabular}
\]

One Nequilibrium:
\[
\begin{tabular}{lcc}
& $C$ & $D$ \\
\cline{2-3}
$A$ & \multicolumn{1}{|c}{$2,2$} & \multicolumn{1}{|c|}{$1,1$} \\
\cline{2-3}
$B$ & \multicolumn{1}{|c}{$1,1$} & \multicolumn{1}{|c|}{$0,0$} \\
\cline{2-3}
\end{tabular}
\]

Two Nequilibriums:
\[
\begin{tabular}{lcc}
& $C$ & $D$ \\
\cline{2-3}
$A$ & \multicolumn{1}{|c}{$2,2$} & \multicolumn{1}{|c|}{$0,0$} \\
\cline{2-3}
$B$ & \multicolumn{1}{|c}{$0,0$} & \multicolumn{1}{|c|}{$2,2$} \\
\cline{2-3}
\end{tabular}
\]
Three Nequilibriums:
\[
\begin{tabular}{lcc}
& $C$ & $D$ \\
\cline{2-3}
$A$ & \multicolumn{1}{|c}{$1,1$} & \multicolumn{1}{|c|}{$0,0$} \\
\cline{2-3}
$B$ & \multicolumn{1}{|c}{$1,1$} & \multicolumn{1}{|c|}{$1,1$} \\
\cline{2-3}
\end{tabular}
\]
Four Nequilibriums:
\[
\begin{tabular}{lcc}
& $C$ & $D$ \\
\cline{2-3}
$A$ & \multicolumn{1}{|c}{$1,1$} & \multicolumn{1}{|c|}{$1,1$} \\
\cline{2-3}
$B$ & \multicolumn{1}{|c}{$1,1$} & \multicolumn{1}{|c|}{$1,1$} \\
\cline{2-3}
\end{tabular}
\]
\end{solution}
\newpage
\section*{Problem 2}
\begin{problem}

\[
\begin{tabular}{lcc}
& $L$ & $R$ \\
\cline{2-3}
$U$ & \multicolumn{1}{|c}{$a,b$} & \multicolumn{1}{|c|}{$c,d$} \\
\cline{2-3}
$D$ & \multicolumn{1}{|c}{$0,0$} & \multicolumn{1}{|c|}{$0,0$} \\
\cline{2-3}
\end{tabular}
 \quad a,c \neq 0, \quad b\neq d\]

Find all Nash equilibria as a function of $a, b, c,$ and $d$, i.e., for each possible combination
of parameter values ($a, b, c,$ and $d$ are the parameters in this game), provide the set of
Nash equilibria.
\begin{solution}
    \[a<0, c<0, b<d \implies (D,L) \cap (D,R)\]
    \[a<0, c<0, b>d \implies (D,L) \cap (D,R)\]
    \[a<0, c>0, b<d \implies (D,L) \cap (U,R)\]
    \[a<0, c>0, b>d \implies (D,L)\]
    \[a>0, c<0, b<d \implies (D,R)\]
    \[a>0, c<0, b>d \implies (U,L) \cap (D,R)\]
    \[a>0, c>0, b<d \implies (U,R)\]
    \[a>0, c>0, b>d \implies (U,L)\]
\end{solution}
\end{problem}
\newpage

\section*{Problem 3}
\begin{problem}
    Consider a three-player strategic game with ordinal preferences, where the three players
simultaneously and independently choose a policy (number) between 0 and 10. The
policy implemented is the median of the three policies chosen. Each player has a most-
preferred policy: player 1 most prefers 0, player 2 most prefers 3, and player 3 most
prefers 10. All players strictly prefer implemented policies closer to their most-preferred
policy to those further away.
\end{problem}

\begin{enumerate}
    \item 
    \begin{problem}
        Show that it is a Nash equilibrium for each player to choose her most-preferred
policy.
    \end{problem}
    \begin{solution}
    Let $x = (0,3,10)$ be the action profile in which the players choose their most preferred policies. We consider the utility functions of players $a,$ $b,$ and $c,$ given that $y$ is the median of the tree choices, then
        \[\mu_1(y) = -y\]
        \[\mu_2(y) = -|y-3|\]
        \[\mu_3(y) = y-10\]
    For the action profile of $x,$ we have that $y_x = 3,$ and thus 
    \[\mu_1(y_x) = -3.\] Consider the other options given to player $1.$ If they choose policy $0,1,2,$ or $3,$ then $y = 3$ still and $\mu_1(y) = -3$. If player $1$ chooses any other policy $z,$ then the median will be $z$ and thus $\mu_1(z) = -z < -3.$ Thus, the best response of player $1$ are policy choices $\{0,1,2,3\}$ given the actions $x_{-1} = (3,10)$ by the other two players.

    Using similar logic, it is easy to show that the best response of player $2$ given $x_{-2} = (0,10)$ is policy $3$ and the best response of player $3$ is choosing policy choices $\{3,4,5,6,7,8,9,10\}.$ 

    Thus, since the action profile $x$ maximizes everybody utility such that for each player $i,$ we have that for any policy $y,$
    \[\mu_i(x) \geq \mu_i(y, x_{-i}),\] then $x$ is a Nash Equlibrium.
    \end{solution}
    

\item 
\begin{problem}
    Is this the only Nash equilibrium? If yes, explain why. If not, provide a counterexample.
\end{problem}
\begin{solution}
    No, it is not the only Nash equilibrium, as the action profile $x = (1,3,9)$ is also in equilibrium, as explained above.
\end{solution}

\end{enumerate}
\newpage
\begin{problem}
    Consider the following generalization of the VHS/Beta game discussed in class. There are infinitely many consumers (formally a continuum), indexed by $i$. Each consumer $i$
chooses VHS or Beta, where the payoff from each action is increasing in the proportion of consumers who have chosen that action. For concreteness, let the payoff from choosing VHS be $\alpha$ and from Beta be $2(1-\alpha)$, where $\alpha$ is the (endogenous) proportion of individuals who have chosen VHS.
\end{problem}
\begin{enumerate}
    \item 
    \begin{problem}
    Show that it is a Nash equilibrium for all consumers to choose VHS. (Hint:
Because consumers are a continuum, a deviation by any one consumer does not
change the proportion $\alpha$ of consumers who have chosen VHS.)
    \end{problem}
    \begin{solution}
    Let $a^\ast$ be the action profile where everybody chooses VHS, that is $\alpha = 1.$ Let $(a, a_{-i})$ be the action profile where $i$ chooses Beta and everybody else chooses VHS. Still, since there are infinitely many players, $\alpha =1.$ Thus, we have that for every player $i:$
    \[\mu(a^\ast) = \alpha = 1, \qquad \mu(a_i, a_{-i}^\ast) = 2(1-\alpha) = 0,\] and thus everybody choosing VHS is a Nash Equilibrium.
    \end{solution}
    \item 
    \begin{problem}
        Show that it is a Nash equilibrium for all consumers to choose Beta.
    \end{problem}
    \begin{solution}
        Let $a^\ast$ be the action profile where all the players choose Beta. Then we have that $\alpha = 0$ and thus $\mu(a^\ast) = 2.$ Then we have that if $(a_i, a_{-i})$ is the action profile where player $i$ chooses VHS, then 
        \[\mu(a^\ast) = 2, \qquad \mu((a_i, a_{-i}^\ast)) = 0,\] and thus $a^\ast$ is a Nash Equilibrium.
    \end{solution}
    \item 
    \begin{problem}
        In addition to the previous two equilibria, there is a third equilibrium in which
some consumers choose VHS and some choose Beta. Find this equilibrium by
deriving the $\alpha$ such that consumers are indifferent between VHS and Beta
    \end{problem}
    \begin{solution}
        When $\alpha = \frac{2}{3}$ (when 2/3 choose Beta, characterized by the action profile $a^\ast$), then suppose that $i$ are the consumers who choose VHS and $j$ be the consumers who choose Beta. Then 
        \[\mu_i(a^\ast)= \alpha = \frac{2}{3}, \quad \mu_j(a^\ast) = 2(1 - \alpha) = \frac{2}{3}.\] Now let $(a_i, a_{i-1}^\ast)$ be the action profile in which a VHS player deviates to Beta while everybody else keeps doing what they were doing, then we get the exact same payoffs as before (a similar reasoning if $j$ deviates), meaning that $\alpha = \frac{2}{3}$ is a weak Nash equilibrium. 
    \end{solution}
\end{enumerate}


\end{document}